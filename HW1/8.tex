\documentclass[11pt]{article}
\usepackage[margin=0.65in, paperwidth=8.5in, paperheight=11in]{geometry}
\usepackage{amsfonts}
\usepackage{amsmath}
\usepackage{amssymb}
\usepackage{bm}
\usepackage{authblk}
\usepackage{graphicx}
\usepackage{listings}
\usepackage{array}
\usepackage{titlesec}
\usepackage{pgfplotstable}

\titlespacing\section{10pt}{14pt plus 4pt minus 2pt}{10pt plus 2pt minus 2pt}
\titlespacing\subsection{0pt}{12pt plus 4pt minus 2pt}{8pt plus 2pt minus 2pt}
\titlespacing\subsubsection{0pt}{12pt plus 4pt minus 2pt}{6pt plus 2pt minus 2pt}
\usepackage{color} %red, green, blue, yellow, cyan, magenta, black, white
\definecolor{mygreen}{RGB}{28,172,0} % color values Red, Green, Blue
\definecolor{mylilas}{RGB}{170,55,241}
\graphicspath{{Final_Project/FEMProject/Results/}}
\title{\bf CSE 401: Numerical Analysis - Fall 2016 \\ Homework 1}
\author{Bhavesh Shrimali (NetID: bshrima2)}
\begin{document}
\maketitle
\section*{Problem 8}
\noindent 
In order to show that the matrix is singular it is sufficient to prove that the determinant is equal to zero.
\begin{align}
\text{ i. e. }
det({\bf A}) = 0
\end{align}
The determinant of the matrix can be computed along the first row
\begin{align*}
& 0.1 \cdot (0.45-0.48) - 0.2 \cdot (0.36-0.42) + 0.3 \cdot (0.32-0.35)\\
 & = 0.003  - 0.012 + 0.009\\
 & = 0  
\end{align*}
Now In order to find out the solution of the system of equations ${\bf Ax = b}$ we proceed with Gaussian-Elimination with partial pivoting
\begin{itemize}
\item $R_2\rightarrow R_2 - 4 \cdot R_1$
\[
\begin{bmatrix}
0.1 & 0.2 & 0.3\\
0 & -0.3 & -0.6\\
0.7 & 0.8 & 0.9
\end{bmatrix}{\bf x}
= 
\begin{Bmatrix}
0.1\\-0.1\\0.5
\end{Bmatrix}
\]
\item $R_3 \rightarrow R_3 - 7 \cdot R_1$
\[
\begin{bmatrix}
0.1 & 0.2 & 0.3\\
0 & -0.3 & -0.6\\
0 & -0.6 & -1.2
\end{bmatrix}{\bf x}
= 
\begin{Bmatrix}
0.1\\-0.1\\-0.2
\end{Bmatrix}
\]
\item $R_3 \rightarrow R_3-2\cdot R_2$
\[
\begin{bmatrix}
0.1 & 0.2 & 0.3\\
0 & -0.3 & -0.6\\
0 & 0 & 0
\end{bmatrix}{\bf x}
= 
\begin{Bmatrix}
0.1\\-0.1\\0
\end{Bmatrix}
\]
\item In essence, we can do another row operation $R_1 \rightarrow R_1 + R_2$
\[
\begin{bmatrix}
0.1 & -0.1 & -0.3\\
0 & -0.3 & -0.6\\
0 & 0 & 0
\end{bmatrix}{\bf x}
= 
\begin{Bmatrix}
0\\-0.1\\0
\end{Bmatrix}
\]
\end{itemize}
Let 
\[ {\bf x } = 
\begin{Bmatrix}
x_1\\
x_2\\
x_3
\end{Bmatrix}
\]
This implies, for 
\begin{align}
\begin{split}
x_3 &= \lambda \\
x_2 &= \frac{1}{3} - 2\lambda\\
x_1 &= \lambda + \frac{1}{3}  \ \ \ \ \ \ \ \ \forall \lambda \in \mathbb{R}
\end{split}
\end{align}
Thus the solution set is described as 
\[ 
\begin{Bmatrix}
x_1\\
x_2\\
x_3
\end{Bmatrix}
= 
\lambda
\begin{Bmatrix}
1\\-2\\1
\end{Bmatrix} 
+ 
\begin{Bmatrix}
1/3\\
1/3\\
0
\end{Bmatrix}
\ \ \ \ \ \forall \lambda \in \mathbb{R}
\]
\subsection*{2}
If Gaussian Elimination with partial pivoting is used to solve the given system the process would eventually fail at {\bf back substitution}
\end{document}