\documentclass[11pt]{article}
\usepackage[margin=0.65in, paperwidth=8.5in, paperheight=11in]{geometry}
\usepackage{amsfonts}
\usepackage{amsmath}
\usepackage{amssymb}
\usepackage{bm}
\usepackage{authblk}
\usepackage{graphicx}
\usepackage{listings}
\usepackage{array}
\usepackage{titlesec}
\usepackage{pgfplotstable}

\titlespacing\section{10pt}{14pt plus 4pt minus 2pt}{10pt plus 2pt minus 2pt}
\titlespacing\subsection{0pt}{12pt plus 4pt minus 2pt}{8pt plus 2pt minus 2pt}
\titlespacing\subsubsection{0pt}{12pt plus 4pt minus 2pt}{6pt plus 2pt minus 2pt}
\usepackage{color} %red, green, blue, yellow, cyan, magenta, black, white
\definecolor{mygreen}{RGB}{28,172,0} % color values Red, Green, Blue
\definecolor{mylilas}{RGB}{170,55,241}
\graphicspath{{Final_Project/FEMProject/Results/}}
\title{\bf CSE 401: Numerical Analysis - Fall 2016 \\ Homework 1}
\author{Bhavesh Shrimali (NetID: bshrima2)}
\begin{document}
\maketitle
\section*{Problem 7}
\noindent 
\subsection*{1}
The determinant of the matrix is given as
\[
\begin{vmatrix}
    1 & 1+{\epsilon} \\
    1-{\epsilon} & 1  
\end{vmatrix}
 = {\epsilon}^2
\]
\subsection*{2}
It can be readily seen from the calculation of the determinant that the corresponding range required on $\epsilon$
\begin{align}
 -\epsilon_{mach} < \epsilon < \epsilon_{mach}
\label{eqn1}
\end{align}
where $\epsilon_{mach}$ denotes the machine epsilon. It can be seen that for all values satisfied by (\ref{eqn1}) the matrix would indeed be stored as 
\[
\begin{bmatrix}
    1 & 1 \\
    1 & 1  
\end{bmatrix}
\]
which is singular. 
\subsection*{3} The L-U Factorization can be readily computed knowing that all the diagonal entries in L are equal to 1.
This permits the following representation for L and U respectively
\[
{\bf L} = 
\begin{bmatrix}
    1 & 0 \\
    \dots & 1  
\end{bmatrix}
\text{ and {\bf U} = }
\begin{bmatrix}
 	\dots & \dots \\
    0 & \dots  
\end{bmatrix}
\]
Now we start with the given matrix ({\bf A}) and try to reduce it to {\bf U}:
\[
{\bf A} = 
\begin{bmatrix}
    1 & 1+{\epsilon} \\
    1-\epsilon & 1  
\end{bmatrix}
\]
Now $R_2 \rightarrow R_2 + (\epsilon -1)\cdot R_1$ which gives, as per rule,
\[
{\bf U} = 
\begin{bmatrix}
1 & 1+\epsilon\\
0 & {\epsilon}^2
\end{bmatrix}
\text{ and } {\bf L }
= 
\begin{bmatrix}
1 &0\\
1-\epsilon &1
\end{bmatrix}
\]
\subsection*{4}
The computed value of {\bf U} would be singular whenever ${\epsilon}^2$ is less than $\epsilon_{mach}$ that is,
\begin{align*}
- \sqrt{\epsilon_{mach}} < \epsilon < \sqrt{\epsilon_{mach}}
\label{eqn3}
\end{align*}
Note: The above inequality holds true for a given {\bf U} i.e. if given a {\bf U} then the only way it can be singular is when (\ref{eqn3}) holds. If epsilon ($\epsilon$) is assumed to be strictly positive then we can ignore the negative part of the inequality, i.e. 
\begin{align}
0 \leq \epsilon < \sqrt{\epsilon_{mach}}
\end{align}
Now if U is reduced from the given {\bf A} then in the very first step when {\bf A} is stored, it is singular if the epsilon($\epsilon$) chosen is such that it is less than the machine epsilon ($\epsilon_{mach}$) and corresponding {\bf U} is also singular. To summarize:
\begin{itemize}
\item If it is the computed {\bf U} as the only thing we need to be concerned about, disregarding any previous calculations, then
\begin{align}
0 \leq \epsilon < \sqrt{\epsilon_{mach}}
\end{align}
\item If we are concerned about determining {\bf U} from the given {\bf A} and all the steps involved therein, then 
\begin{align}
0 \leq \epsilon < \epsilon_{mach}
\end{align}
Because if $\epsilon$ is less than $\epsilon_{mach}$ then {\bf A} becomes singular and correspondingly {\bf U} obtained from {\bf A} is singular
\end{itemize}
\end{document}