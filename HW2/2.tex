\documentclass[11pt]{article}
\usepackage[margin=0.65in, paperwidth=8.5in, paperheight=11in]{geometry}
\usepackage{amsfonts}
\usepackage{amsmath}
\usepackage{amssymb}
\usepackage{bm}
\usepackage{authblk}
\usepackage{graphicx}
\usepackage{listings}
\usepackage{array}
\usepackage{titlesec}
\usepackage{pgfplotstable}

\titlespacing\section{10pt}{14pt plus 4pt minus 2pt}{10pt plus 2pt minus 2pt}
\titlespacing\subsection{0pt}{12pt plus 4pt minus 2pt}{8pt plus 2pt minus 2pt}
\titlespacing\subsubsection{0pt}{12pt plus 4pt minus 2pt}{6pt plus 2pt minus 2pt}
\usepackage{color} %red, green, blue, yellow, cyan, magenta, black, white
\definecolor{mygreen}{RGB}{28,172,0} % color values Red, Green, Blue
\definecolor{mylilas}{RGB}{170,55,241}
\graphicspath{{Final_Project/FEMProject/Results/}}
\title{\bf CSE 401: Numerical Analysis - Fall 2016 \\ Homework 2}
\author{Bhavesh Shrimali (NetID: bshrima2)}
\begin{document}
\maketitle
\section*{Problem 2:}
\noindent Given system: 
$$ f(t,{\bf x}) = x_1 \cdot t + x_2 \cdot e^t $$
and corresponding data ($t_i$,$f_i$)
\[
\begin{Bmatrix}
t_t\\t_2\\t_3
\end{Bmatrix}
=
\begin{Bmatrix}
1\\2\\3
\end{Bmatrix}
\]
\[
\begin{Bmatrix}
f_1\\
f_2\\
f_3
\end{Bmatrix}
=
\begin{Bmatrix}
2\\3\\5
\end{Bmatrix}
\]
Thus the linear least squares system can be set up as follows:
\[
\begin{bmatrix}
1 & e \\
2 & e^2 \\
3 & e^3
\end{bmatrix}
\begin{Bmatrix}
x_1\\x_2
\end{Bmatrix}
=
\begin{bmatrix}
2\\3\\5
\end{bmatrix}
\]
\end{document}