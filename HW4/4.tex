\ProvidesPackage{commands}
\documentclass[11pt]{article}
\usepackage{epstopdf}
\usepackage{subfigure,graphicx}
\usepackage{amsmath}
\usepackage{epsf}
\usepackage{amsfonts}
\usepackage{amssymb}
\usepackage{color}
\usepackage{mathtools}
\usepackage{placeins}
\usepackage{booktabs}
\usepackage{enumitem}
\usepackage{caption}
\usepackage[margin=0.8in, paperwidth=8.5in, paperheight=11in]{geometry}
\usepackage{amsfonts}
\usepackage{amsmath}
\usepackage{amsbsy}
\usepackage{authblk}
\usepackage{graphicx}
\usepackage{listings}
\usepackage{array}
\usepackage{titlesec}
\usepackage{amssymb}
\usepackage{bm}
\usepackage{mathtools}
\usepackage{titlesec}

\usepackage[latin1]{inputenc}\newcommand{\bs}[1]{\boldsymbol{#1}}
\newcommand{\del}[2]{\frac{\partial {#1}}{\partial {#2}}}
\newcommand{\D}[2]{\frac{D^{\overline{\alpha}}}{\overline{\alpha !}}{#1}(#2,#2)\ {\bf x}^{\overline{\alpha}}}
\newcommand{\dv}[3]{\frac{{\rm d}^{#1}{#2}}{d{#3}^{#1}}}
\newcommand{\ddel}[5]{\frac{\partial^{ {#1} + {#2}} {#3}}{\partial {#4}^{#1} \partial{#5}^{#2}}}
\newcommand{\dev}{{\rm {\bf dev}}}
\newcommand{\proj}[1]{\frac{1}{R^2}{\bf X}\otimes{\bf X}}
\newcommand{\Ie}[1]{I^{\rm e}_{#1}}
\newcommand{\Ce}[1]{\bf C^{\rm e^{#1}}}
\newcommand{\Fe}[2]{F^{\rm e^{#2}}_{#1}}
\newcommand{\Fv}[2]{F^{\rm v^{#2}}_{#1}}
\newcommand{\f}[2]{f^{\rm {#2}}_{#1}}
\newcommand{\fv}[2]{f^{\rm v^{#2}}_{#1}}
\newcommand{\dfv}[2]{\dot{f}^{\rm v^{#2}}_{#1}}
\newcommand{\tGam}[2]{\tilde{\Gamma}^{\rm v^{#2}}_{#1}}
\newcommand{\Gam}[2]{\Gamma^{\rm v^{#2}}_{#1}}
\newcommand{\A}[1]{\mathcal{A}_{#1}}
\newcommand{\F}[2]{F^{\rm #2}_{#1}}
\newcommand{\hpeq}{\hat{\psi}^{\rm Eq}}
\newcommand{\hpneq}{\hat{\psi}^{\rm NEq}}
\newcommand{\etak}{\eta_K({I_1,I_2,J},{\bf C^{\rm e}, B^{\rm v}})}
\newcommand{\nuk}{\nu_K({I_1,I_2,J},{\bf C^{\rm e}, B^{\rm v}})}
\newcommand{\thetak}{\theta_K({I_1,I_2,J},{\bf C^{\rm e}, B^{\rm v}})}
\newcommand{\etaj}{\eta_J({I_1,I_2,J},{\bf C^{\rm e}, B^{\rm v}})}
\newcommand{\dFv}[2]{\dot{F}^{\rm v^{#2}}_{#1}}
\newcommand{\hatpsi}{\widehat{\psi}(I_1, I_2,I^{\rm e}_1,I^{\rm e}_2,J)}
\newcommand{\hpsi}{\widehat{\psi}(I_1,I^{\rm e}_1,J)}
\newcommand{\Fh}[1]{\widehat{\mathcal{F}}\left({\bf F, \Fv{}{}}, {#1}\right)}
\newcommand{\Fhstar}[1]{\widehat{\mathcal{F}}^*\left({\bf F, \Fv{}{}}, {#1}\right)}
\newcommand{\sbar}{\overline{\bm{\sigma}}}
\newcommand{\hpsicomp}[1]{\sum_{r=1}^{2}\left\{\frac{3^{1-\alpha_r}}{2\alpha_r}\mu_r(I^{\alpha_r}_1-3^{\alpha_r})
+\frac{3^{1-a_r}}{2a_r}m_r({\Ie{1}}^{^{a_r}}-3^{a_r})\right\}
+\mu{#1}+\kappa{#1}^2}
\newcommand{\Ni}[1]{N^{(e)}_i(#1)}
\newcommand{\hNi}[1]{\hat{{N}}^{(e)}_i(#1)}
\newcommand{\Ld}{L^{\dagger}}
\newcommand{\intinfinf}{\int_{-\infty}^{\infty} \int_{-\infty}^{\infty}}
\newcommand{\LLnorm}[1]{\left\lVert{#1}\right\rVert_2}
\newcommand{\Linorm}[1]{{\left\lVert{#1}\right\rVert_\infty}}
\newcommand{\tr}{\rm tr}
\newcommand{\deldel}[2]{\frac{\partial^2 {#1}}{\partial {#2}^2}}
\newcommand{\kd}[1]{\delta_{#1}}



\titlespacing\section{10pt}{10pt plus 4pt minus 2pt}{10pt plus 2pt minus 2pt}
\titlespacing\subsection{0pt}{8pt plus 4pt minus 2pt}{8pt plus 2pt minus 2pt}
\titlespacing\subsubsection{0pt}{12pt plus 4pt minus 2pt}{6pt plus 2pt minus 2pt}
\titlespacing*{\title}{-2ex}{*-2ex}{-2ex}
\usepackage{color} %red, green, blue, yellow, cyan, magenta, black, white
\definecolor{mygreen}{RGB}{28,172,0} % color values Red, Green, Blue
\definecolor{mylilas}{RGB}{170,55,241}
\setlength\parindent{0pt}
\graphicspath{{/}}

\title{\bf CSE 401: Numerical Analysis \\ HW 4}
\author{Bhavesh Shrimali \\ NetID: bshrima2}
\date{\today}
\titlespacing*{\title}{-2ex}{*-2ex}{-2ex}
\begin{document}
\maketitle \hrule \hrule \hrule
\section*{Solution 4:}
Given function
\[
f(x,y)
=
\frac{1}{2}(x^2_1 - x_2)^2 + \frac{1}{2} (1-x_1)^2
\]
\subsection*{(a): }
The critical points of the given function are given by 
\begin{align*}
\bm\nabla f = 
\begin{Bmatrix}
x_1 - 2x_1(x_2-x^2_1) -1 \\
x_2 - x^2_1
\end{Bmatrix} = \begin{Bmatrix}
0\\0
\end{Bmatrix}\implies (x_1,x_2) = \begin{bmatrix}
\left\{
1,1
\right\}
\end{bmatrix}
\end{align*}
The hessian matrix, $\bf{H}(x,y)$, is given by
\begin{align*}
\begin{bmatrix}
6x^2_1 - 2x_2 + 1 & -2x_1\\
-2x_1 & 1
\end{bmatrix}
\end{align*}
The hessian matrix evaluated at the critical point $(1,1)$ is
\[
{\bf H} (1,1)
=
\begin{bmatrix}
5 & -2 \\
-2 & 1
\end{bmatrix}
\]
The corresponding eigenvalues are given by
\begin{align*}
\begin{vmatrix}
5-\lambda & -2 \\
-2 & 1-\lambda
\end{vmatrix} = 0 \implies \lambda_1 = 0.1716\ ; \ \ \ \lambda_2 = 5.8284
\end{align*}
Thus the hessian matrix is positive definite implying that the point $(1,1)$ is a local minimum. \\ \hrule
\subsection*{(b): }
The corresponding iteration is described as follows: 
\begin{align*}
{\bf H}_f({\bf x}_k) s_k = -\nabla f({\bf x}_k) \\
-\nabla f({\bf x}_k) = 
\begin{bmatrix}
-9\\2
\end{bmatrix}\ ; \ \ \ \ {\bf H}_f({\bf x}_k) = 
\begin{bmatrix}
21 & -4 \\ -4 & 1
\end{bmatrix}
\end{align*}
which gives
\begin{align*}
s_k
=
\begin{bmatrix}
-0.2\\1.2
\end{bmatrix}
\end{align*}
thus 
\begin{align*}
x_{k+1}
=
\begin{Bmatrix}
1.8\\3.2
\end{Bmatrix}
\end{align*}\hrule
\subsection*{(c): }
In order to see if it is a good step, we compute 
\begin{align*}
\nabla f({\bf x}_k)^Ts_k
= -4.2 < 0
\end{align*}
The vector obtained after the first iteration does come close to the true solution in one of the components. Hence it is a good step in the sense of descent direction. \\ \hrule 
\subsection*{(d): }
While we know that the exact solution to the problem is $x = [1.0,\  1.0]^T$, the last iteration in Newton's method, though bringing $x_1$ close to it's exact solution, takes $x_2$ farther than the initial guess. Hence in this sense it is a bad step. We can also compute the Euclidean norm of the difference between the true solution and current iterate. 
\begin{align*}
||x_1 - x||_{2}
& =
\sqrt{2}\\
||x_2 - x||_{2}
& =
\sqrt{5.48}
\end{align*}
which clearly indicates that the euclidean norm of the distance increases. Hence it is a bad step. \\ \hrule \hrule \hrule 
\end{document}