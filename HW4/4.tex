\input{commands}
\title{\bf CSE 401: Numerical Analysis \\ HW 4}
\author{Bhavesh Shrimali \\ NetID: bshrima2}
\date{\today}
\titlespacing*{\title}{-2ex}{*-2ex}{-2ex}
\begin{document}
\maketitle \hrule \hrule \hrule
\section*{Solution 4:}
Given function
\[
f(x,y)
=
\frac{1}{2}(x^2_1 - x_2)^2 + \frac{1}{2} (1-x_1)^2
\]
\subsection*{(a): }
The critical points of the given function are given by 
\begin{align*}
\bm\nabla f = 
\begin{Bmatrix}
x_1 - 2x_1(x_2-x^2_1) -1 \\
x_2 - x^2_1
\end{Bmatrix} = \begin{Bmatrix}
0\\0
\end{Bmatrix}\implies (x_1,x_2) = \begin{bmatrix}
\left\{
1,1
\right\}
\end{bmatrix}
\end{align*}
The hessian matrix, $\bf{H}(x,y)$, is given by
\begin{align*}
\begin{bmatrix}
6x^2_1 - 2x_2 + 1 & -2x_1\\
-2x_1 & 1
\end{bmatrix}
\end{align*}
The hessian matrix evaluated at the critical point $(1,1)$ is
\[
{\bf H} (1,1)
=
\begin{bmatrix}
5 & -2 \\
-2 & 1
\end{bmatrix}
\]
The corresponding eigenvalues are given by
\begin{align*}
\begin{vmatrix}
5-\lambda & -2 \\
-2 & 1-\lambda
\end{vmatrix} = 0 \implies \lambda_1 = 0.1716\ ; \ \ \ \lambda_2 = 5.8284
\end{align*}
Thus the hessian matrix is positive definite implying that the point $(1,1)$ is a local minimum. \\ \hrule
\subsection*{(b): }
The corresponding iteration is described as follows: 
\begin{align*}
{\bf H}_f({\bf x}_k) s_k = -\nabla f({\bf x}_k) \\
-\nabla f({\bf x}_k) = 
\begin{bmatrix}
-9\\2
\end{bmatrix}\ ; \ \ \ \ {\bf H}_f({\bf x}_k) = 
\begin{bmatrix}
21 & -4 \\ -4 & 1
\end{bmatrix}
\end{align*}
which gives
\begin{align*}
s_k
=
\begin{bmatrix}
-0.2\\1.2
\end{bmatrix}
\end{align*}
thus 
\begin{align*}
x_{k+1}
=
\begin{Bmatrix}
1.8\\3.2
\end{Bmatrix}
\end{align*}\hrule
\subsection*{(c): }
In order to see if it is a good step, we compute 
\begin{align*}
\nabla f({\bf x}_k)^Ts_k
= -4.2 < 0
\end{align*}
The vector obtained after the first iteration does come close to the true solution in one of the components. Hence it is a good step in the sense of descent direction. \\ \hrule 
\subsection*{(d): }
While we know that the exact solution to the problem is $x = [1.0,\  1.0]^T$, the last iteration in Newton's method, though bringing $x_1$ close to it's exact solution, takes $x_2$ farther than the initial guess. Hence in this sense it is a bad step. We can also compute the Euclidean norm of the difference between the true solution and current iterate. 
\begin{align*}
||x_1 - x||_{2}
& =
\sqrt{2}\\
||x_2 - x||_{2}
& =
\sqrt{5.48}
\end{align*}
which clearly indicates that the euclidean norm of the distance increases. Hence it is a bad step. \\ \hrule \hrule \hrule 
\end{document}