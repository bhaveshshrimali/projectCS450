\input{commands}
\title{\bf CSE 401: Numerical Analysis \\ HW 4}
\author{Bhavesh Shrimali \\ NetID: bshrima2}
\date{\today}
\titlespacing*{\title}{-2ex}{*-2ex}{-2ex}
\begin{document}
\maketitle \hrule \hrule \hrule
\section*{Solution 3:}
\subsection*{(a):}
Given function:
\[
f(x,y)
=
x^2 - 4xy + y^2
\]
We compute the critical points of the given function: 
\begin{align*}
\bm\nabla f
=
\begin{Bmatrix}
2x - 4y \\
-4x + 2y
\end{Bmatrix}
=
\begin{Bmatrix}
0\\0
\end{Bmatrix}\implies x = 2y\ ; \ \ \ y = 2x \implies x=y=0
\end{align*}
Computing the Hessian Matrix at the critical point $(0,0)$
\begin{align*}
\begin{bmatrix}
2 & -4 \\
-4 & 2
\end{bmatrix}
\end{align*}
Computing the eigenvalues of the hessian matrix
\begin{align*}
\begin{vmatrix}
2-\lambda & -4 \\
-4 & 2-\lambda
\end{vmatrix} = 0 \implies \lambda_1 = -2\ ; \ \ \ \lambda_2 = 6
\end{align*}
The hessian matrix is non-singular but has both negative and positive eigen-values, thus it is indefinite. {\bf Therefore,  $(0,0)$ is a saddle point of $f(x,y)$}. 
\subsubsection*{Comments on Global Maximum and Minimum: }
We can infer that
\begin{itemize}
\item The given function does not have any global maxima or minima. In other words the function is {\bf non-coercive}
\item This can be observed by taking the direction $x=-y$, as along this direction the function shoots off to $\infty$ as $||y||\rightarrow \infty$ and similarly shoots off, along $x=y$, to $-\infty$ as $||y||\rightarrow \infty$ 
\end{itemize}\hrule 
\subsection*{(b): }
Given function:
\[
f(x,y)
=
x^4 - 4xy + y^4
\]
We compute the critical points of the given function: 
\begin{align*}
\bm\nabla f
=
\begin{Bmatrix}
4x^3 - 4y \\
-4x + 4y^3
\end{Bmatrix}
=
\begin{Bmatrix}
0\\0
\end{Bmatrix}\implies x^3 = y\ ; \ \ \ y^3 = x \implies (x,y) = \begin{bmatrix}
\left\{0,0 \right\}, & \left\{1,1 \right\}, & \left\{-1,-1 \right\} 
\end{bmatrix}
\end{align*}
Computing the Hessian Matrix ${\bf H}(x,y)$
\begin{align*}
{\bf H}(x,y) = 
\begin{bmatrix}
12x^2 & -4 \\
-4 & 12y^2
\end{bmatrix}
\end{align*}
Now we evaluate the Hessian Matrix at all the critical points.
\subsubsection*{$\bm\cdot$ H(0,0)}
\begin{align*}
{\bf H} (0,0) =
\begin{bmatrix}
0 & -4 \\
-4 & 0
\end{bmatrix}
\end{align*}
Computing the eigenvalues of the hessian matrix
\begin{align*}
\begin{vmatrix}
-\lambda & -4 \\
-4 & -\lambda
\end{vmatrix} = 0 \implies \lambda_1 = -4\ ; \ \ \ \lambda_2 = 4
\end{align*}
The hessian matrix is non-singular but has both negative and positive eigen-values, thus it is indefinite. {\bf Therefore, $(0,0)$ is a saddle point of $f(x,y)$.} 
\subsubsection*{$\bm\cdot$ H(1,1) and H(-1,-1)}
\begin{align*}
{\bf H} (1,1) = {\bf H} (-1,-1)
\begin{bmatrix}
12 & -4 \\
-4 & 12
\end{bmatrix}
\end{align*}
Computing the eigenvalues of the hessian matrix
\begin{align*}
\begin{vmatrix}
12-\lambda & -4 \\
-4 & 12-\lambda
\end{vmatrix} = 0 \implies \lambda_1 = 8\ ; \ \ \ \lambda_2 = 16
\end{align*}
The hessian matrix is non-singular and has only positive eigen-values, thus it is positive-definite. {\bf Therefore, $(1,1)$ and $(-1,-1)$ are local minima of $f(x,y)$ at least. We will check now if they are global minima or not} 
\subsubsection*{Comments on Global Maximum and Minimum: }
We can infer that
\begin{itemize}
\item The given function has two global minima. This is because the function is {\bf coercive}. The leading order term $x^4+y^4$ always dominates the third term and shoots off to $\infty$ as $||x,y||\rightarrow (\infty)$ or $-\infty$, and hence the function has two global minima and no global maxima.  
\item $f(1,1) = -2$ and $f(-1,-1) = -2$, thus both are global minima. Note that it is not possible to find any global maxima. 
\item We can see this by taking the direction $x=-y$, which shoots off to $\infty$ as $||x||\rightarrow\infty$
\end{itemize}\hrule 
\subsection*{(c): }
Given function:
\[
f(x,y)
=
2x^3 - 3x^2 - 6xy(x-y-1)
\]
We compute the critical points of the given function: 
\begin{align*}
\bm\nabla f
=
\begin{Bmatrix}
 6y(y - x + 1) - 6xy - 6x + 6x^2 \\
6xy + 6x(y - x + 1)
\end{Bmatrix}
=
\begin{Bmatrix}
0\\0
\end{Bmatrix}\implies (x,y) = \begin{bmatrix}
\left\{0,0 \right\}, & \left\{0,-1 \right\}, & \left\{-1,-1 \right\},\left\{1,0 \right\} 
\end{bmatrix}
\end{align*}
Computing the Hessian Matrix ${\bf H}(x,y)$
\begin{align*}
{\bf H}(x,y) = 
\begin{bmatrix}
12x-12y-6 & 12y-12x+6 \\
12y-12x+6 & 12x
\end{bmatrix}
\end{align*}
Now we evaluate the Hessian Matrix at all the critical points.
\subsubsection*{$\bm\cdot$ H(0,0)}
\begin{align*}
{\bf H} (0,0) =
\begin{bmatrix}
-6 & 6 \\
6 & 0
\end{bmatrix}
\end{align*}
Computing the eigenvalues of the hessian matrix
\begin{align*}
\begin{vmatrix}
-6-\lambda & 6 \\
6 & -\lambda
\end{vmatrix} = 0 \implies \lambda_1 = -9.7082\ ; \ \ \ \lambda_2 = 3.7082
\end{align*}
The hessian matrix is non-singular but has both negative and positive eigen-values, thus it is indefinite. {\bf Therefore, $(0,0)$ is a saddle point of $f(x,y)$}. 
\subsubsection*{$\bm\cdot$ H(-1,-1)}
\begin{align*}
{\bf H} (-1,-1) =
\begin{bmatrix}
-6 & 6 \\
6 & -12
\end{bmatrix}
\end{align*}
Computing the eigenvalues of the hessian matrix
\begin{align*}
\begin{vmatrix}
-6-\lambda & 6 \\
6 & -12-\lambda
\end{vmatrix} = 0 \implies \lambda_1 = -15.7082\ ; \ \ \ \lambda_2 = -2.2918
\end{align*}
The hessian matrix is non-singular and has only negative eigen-values, thus it is negative-definite. {\bf Therefore, $(-1,-1)$ is a local maximum of $f(x,y)$.} 
\subsubsection*{$\bm\cdot$ H(0,-1)}
\begin{align*}
{\bf H} (0,-1) = 
\begin{bmatrix}
6 & -6 \\
-6 & 0
\end{bmatrix}
\end{align*}
Computing the eigenvalues of the hessian matrix
\begin{align*}
\begin{vmatrix}
6-\lambda & -6 \\
-6 & -\lambda
\end{vmatrix} = 0 \implies \lambda_1 = -3.7082\ ; \ \ \ \lambda_2 = 9.7082
\end{align*}
The hessian matrix is non-singular and has both positive and negative eigen-values, thus it is indefinite. { \bf Therefore, $(0,-1)$ is a saddle point of $f(x,y)$.} 
\subsubsection*{$\bm\cdot$ H(1,0)}
\begin{align*}
{\bf H} (1,0) = 
\begin{bmatrix}
6 & -6 \\
-6 & 12
\end{bmatrix}
\end{align*}
Computing the eigenvalues of the hessian matrix
\begin{align*}
\begin{vmatrix}
6-\lambda & -6 \\
-6 & 12-\lambda
\end{vmatrix} = 0 \implies \lambda_1 = 2.2918\ ; \ \ \ \lambda_2 = 15.7082
\end{align*}
The hessian matrix is non-singular and has positive eigen-values, thus it is positive-definite. {\bf Therefore, $(1,0)$ is a local minimum of $f(x,y)$.} 
\subsubsection*{Comments on Global Maximum and Minimum: }
We can infer that
\begin{itemize}
\item The given function has two saddle points and one each --- a local maximum and local minimum. 
\item Let us observe the function along the curve $y=x-1$. Along this curve $f(x,y(x)) = 2x^3 - 3x^2$. This curve, at $x\rightarrow\infty$, shoots off to $\infty$ and, at $x\rightarrow-\infty$, shoots off to $-\infty$. Therefore there are no global maxima/minima. In other words the function is {\bf non-coercive}
\end{itemize}\hrule 
\subsection*{(d): }
Given function:
\[
f(x,y)
=
(x-y)^4 + x^2 - y^2 -2x + 2y + 1
\]
We compute the critical points of the given function: 
\begin{align*}
\bm\nabla f
=
\begin{Bmatrix}
2x+4(x-y)^3-2 \\
2-4(x-y)^3-2y
\end{Bmatrix}
=
\begin{Bmatrix}
0\\0
\end{Bmatrix}\implies (x,y) = \begin{bmatrix}
\left\{1,1 \right\} 
\end{bmatrix}
\end{align*}
Computing the Hessian Matrix ${\bf H}(x,y)$
\begin{align*}
{\bf H}(x,y) = 
\begin{bmatrix}
12(x-y)^2+2 & -12(x-y)^2 \\
-12(x-y)^2 & 12(x-y)^2-2
\end{bmatrix}
\end{align*}
Now we evaluate the Hessian Matrix at all the critical points.
\subsubsection*{$\bm\cdot$ H(1,1)}
\begin{align*}
{\bf H} (1,1) =
\begin{bmatrix}
2 & 0 \\
0 & -2
\end{bmatrix}
\end{align*}
Computing the eigenvalues of the hessian matrix
\begin{align*}
\begin{vmatrix}
2-\lambda & 0 \\
0 & -2-\lambda
\end{vmatrix} = 0 \implies \lambda_1 = -2\ ; \ \ \ \lambda_2 = 2
\end{align*}
The hessian matrix is non-singular but has both negative and positive eigen-values, thus it is indefinite. {\bf Therefore, $(1,1)$ is a saddle point of $f(x,y)$.} 
\subsubsection*{Comments on Global Maximum and Minimum: }
We can infer that
\begin{itemize}
\item The given function has one saddle point. 
\item Since the leading term in the function is $(x-y)^4$, the function has no global maximum. This is illustrated via the following figure. As we can see that the function is increasing towards the bottom left corner $y\rightarrow\infty$. 
\item We can also consider the function along $x=y$, here $f(x,y(x)) = 1$ and thus $f(x,y)$ has no global minimizer. In other words the function is {\bf non-coercive}. 
\end{itemize}
\begin{center}
\includegraphics[width=7.in]{plot4} \\
\end{center}\hrule \hrule \hrule 
\end{document}