\ProvidesPackage{commands}
\documentclass[11pt]{article}
\usepackage{epstopdf}
\usepackage{subfigure,graphicx}
\usepackage{amsmath}
\usepackage{epsf}
\usepackage{amsfonts}
\usepackage{amssymb}
\usepackage{color}
\usepackage{mathtools}
\usepackage{placeins}
\usepackage{booktabs}
\usepackage{enumitem}
\usepackage{caption}
\usepackage[margin=0.8in, paperwidth=8.5in, paperheight=11in]{geometry}
\usepackage{amsfonts}
\usepackage{amsmath}
\usepackage{amsbsy}
\usepackage{authblk}
\usepackage{graphicx}
\usepackage{listings}
\usepackage{array}
\usepackage{titlesec}
\usepackage{amssymb}
\usepackage{bm}
\usepackage{mathtools}
\usepackage{titlesec}

\usepackage[latin1]{inputenc}\newcommand{\bs}[1]{\boldsymbol{#1}}
\newcommand{\del}[2]{\frac{\partial {#1}}{\partial {#2}}}
\newcommand{\D}[2]{\frac{D^{\overline{\alpha}}}{\overline{\alpha !}}{#1}(#2,#2)\ {\bf x}^{\overline{\alpha}}}
\newcommand{\dv}[3]{\frac{{\rm d}^{#1}{#2}}{d{#3}^{#1}}}
\newcommand{\ddel}[5]{\frac{\partial^{ {#1} + {#2}} {#3}}{\partial {#4}^{#1} \partial{#5}^{#2}}}
\newcommand{\dev}{{\rm {\bf dev}}}
\newcommand{\proj}[1]{\frac{1}{R^2}{\bf X}\otimes{\bf X}}
\newcommand{\Ie}[1]{I^{\rm e}_{#1}}
\newcommand{\Ce}[1]{\bf C^{\rm e^{#1}}}
\newcommand{\Fe}[2]{F^{\rm e^{#2}}_{#1}}
\newcommand{\Fv}[2]{F^{\rm v^{#2}}_{#1}}
\newcommand{\f}[2]{f^{\rm {#2}}_{#1}}
\newcommand{\fv}[2]{f^{\rm v^{#2}}_{#1}}
\newcommand{\dfv}[2]{\dot{f}^{\rm v^{#2}}_{#1}}
\newcommand{\tGam}[2]{\tilde{\Gamma}^{\rm v^{#2}}_{#1}}
\newcommand{\Gam}[2]{\Gamma^{\rm v^{#2}}_{#1}}
\newcommand{\A}[1]{\mathcal{A}_{#1}}
\newcommand{\F}[2]{F^{\rm #2}_{#1}}
\newcommand{\hpeq}{\hat{\psi}^{\rm Eq}}
\newcommand{\hpneq}{\hat{\psi}^{\rm NEq}}
\newcommand{\etak}{\eta_K({I_1,I_2,J},{\bf C^{\rm e}, B^{\rm v}})}
\newcommand{\nuk}{\nu_K({I_1,I_2,J},{\bf C^{\rm e}, B^{\rm v}})}
\newcommand{\thetak}{\theta_K({I_1,I_2,J},{\bf C^{\rm e}, B^{\rm v}})}
\newcommand{\etaj}{\eta_J({I_1,I_2,J},{\bf C^{\rm e}, B^{\rm v}})}
\newcommand{\dFv}[2]{\dot{F}^{\rm v^{#2}}_{#1}}
\newcommand{\hatpsi}{\widehat{\psi}(I_1, I_2,I^{\rm e}_1,I^{\rm e}_2,J)}
\newcommand{\hpsi}{\widehat{\psi}(I_1,I^{\rm e}_1,J)}
\newcommand{\Fh}[1]{\widehat{\mathcal{F}}\left({\bf F, \Fv{}{}}, {#1}\right)}
\newcommand{\Fhstar}[1]{\widehat{\mathcal{F}}^*\left({\bf F, \Fv{}{}}, {#1}\right)}
\newcommand{\sbar}{\overline{\bm{\sigma}}}
\newcommand{\hpsicomp}[1]{\sum_{r=1}^{2}\left\{\frac{3^{1-\alpha_r}}{2\alpha_r}\mu_r(I^{\alpha_r}_1-3^{\alpha_r})
+\frac{3^{1-a_r}}{2a_r}m_r({\Ie{1}}^{^{a_r}}-3^{a_r})\right\}
+\mu{#1}+\kappa{#1}^2}
\newcommand{\Ni}[1]{N^{(e)}_i(#1)}
\newcommand{\hNi}[1]{\hat{{N}}^{(e)}_i(#1)}
\newcommand{\Ld}{L^{\dagger}}
\newcommand{\intinfinf}{\int_{-\infty}^{\infty} \int_{-\infty}^{\infty}}
\newcommand{\LLnorm}[1]{\left\lVert{#1}\right\rVert_2}
\newcommand{\Linorm}[1]{{\left\lVert{#1}\right\rVert_\infty}}
\newcommand{\tr}{\rm tr}
\newcommand{\deldel}[2]{\frac{\partial^2 {#1}}{\partial {#2}^2}}
\newcommand{\kd}[1]{\delta_{#1}}



\titlespacing\section{10pt}{10pt plus 4pt minus 2pt}{10pt plus 2pt minus 2pt}
\titlespacing\subsection{0pt}{8pt plus 4pt minus 2pt}{8pt plus 2pt minus 2pt}
\titlespacing\subsubsection{0pt}{12pt plus 4pt minus 2pt}{6pt plus 2pt minus 2pt}
\titlespacing*{\title}{-2ex}{*-2ex}{-2ex}
\usepackage{color} %red, green, blue, yellow, cyan, magenta, black, white
\definecolor{mygreen}{RGB}{28,172,0} % color values Red, Green, Blue
\definecolor{mylilas}{RGB}{170,55,241}
\setlength\parindent{0pt}
\graphicspath{{/}}

\title{\bf CSE 401: Numerical Analysis \\ HW 3}
\author{Bhavesh Shrimali \\ NetID: bshrima2}
\date{\today}
\titlespacing*{\title}{-2ex}{*-2ex}{-2ex}
\begin{document}
\maketitle \hrule \hrule \hrule
\section*{Solution 2:}
Given Matrix: 
\[
{\bf A} = 
\begin{bmatrix}
1 & 4 \\
1 & 1 
\end{bmatrix}
\]
\subsection*{(a):}The characteristic polynomial of ${\bf A}$ is given by: 
\begin{align*}
\det ({\bf A} - \lambda {\bf I}) = 
\begin{vmatrix}
1-\lambda & 4 \\
1 & 1-\lambda
\end{vmatrix} = 0 \\
\implies (\lambda - 1)^2 - 4 = 0\\
\lambda^2 - 2\lambda - 3 = 0
\end{align*}\hrule
\subsection*{(b):}
Roots of the characteristic polynomial are
\[
\lambda = -1 ,3 
\]\hrule
\subsection*{(c):}
The eigenvalues of the matrix ${\bf A}$ are the roots of the characteristic polynomial, i.e. $\lambda_1 = -1,\ \lambda_2 = 3$ \\ \hrule
\subsection*{(d):}
The Eigen-vectors of $\bf A$ are calculated as follows:
\subsubsection*{$\cdot$ Corresponding to $\lambda = -1$:}
Note that the eigen-vectors, in this case, have been normalized in the $L^\infty$ norm. 
\begin{align*}
& ({\bf A} + {\bf I})\ {\bf x} = {\bf 0} \\
& \begin{bmatrix}
2 & 4 \\
1 & 2
\end{bmatrix}
\begin{Bmatrix}
\phi_1 \\ \phi_2
\end{Bmatrix} = {\bf 0}\ ;\ \ \ \text{also}\ \ \ \max_i{\phi_i} = 1 \\ \\
& \begin{Bmatrix}
\phi_1 \\ \phi_2
\end{Bmatrix} = 
\begin{Bmatrix}
1	\\ -0.5
\end{Bmatrix}
\end{align*}
\subsubsection*{$\cdot$ Corresponding to $\lambda = 3$}
\begin{align*}
& ({\bf A} -3 {\bf I})\ {\bf x} = {\bf 0} \\
& \begin{bmatrix}
-2 & 4 \\
1 & -2
\end{bmatrix}
\begin{Bmatrix}
\phi_3 \\ \phi_4
\end{Bmatrix} = {\bf 0}\ ;\ \ \ \text{also}\ \ \ \max_i \phi_i = 1 \\ \\
& \begin{Bmatrix}
\phi_3 \\ \phi_4
\end{Bmatrix} = 
\begin{Bmatrix}
1 \\ 0.5
\end{Bmatrix}
\end{align*}\hrule
\subsection*{(e):}
Given Vector
\[{\bf x}_0
\begin{Bmatrix}
1 \\ 1
\end{Bmatrix}
\]
Thus, the first iteration of the power iteration is as follows. Note that in this particular case, we normalize the Eigen-vector obtained below, in the $L^\infty$ norm, and hence 
\begin{align*}
& {\bf A}{\bf x}_0
={\bf x}_1 = 
\begin{Bmatrix}
1 \\ 0.4
\end{Bmatrix}
\end{align*}\hrule
\subsection*{(f): }
The power iteration will converge to the eigen-vector corresponding to the dominant eigenvalue of $\bf A$ ($\lambda = 3$), which is shown as below
\begin{align*}
& {\bf A}{\bf x}_1
={\bf x}_2 = {\bf x}_2/{\Linorm{{\bf x}_2}} = \frac{1}{2.6}
\begin{Bmatrix}
2.6 \\ 1.4
\end{Bmatrix}
= \begin{Bmatrix}
1\\0.54
\end{Bmatrix} \\ \\
& {\bf A}{\bf x}_2
={\bf x}_3/{\Linorm{{\bf x}_3}} = \frac{1}{3.16}
\begin{Bmatrix}
3.16 \\ 1.54
\end{Bmatrix}= \begin{Bmatrix}
1\\0.487
\end{Bmatrix}
\end{align*}
Thus we can easily see that the eigenvector is converging to the one, corresponding to $\lambda = 3$ as obtained in the part (d). Here the procedure of normalization has been carried out, to implement power-iteration, with respect to the $L^\infty$ norm of the vector. The same could be carried out using the $L^2$ norm as well, just that the $L^\infty$ norm is less expensive to determine. \\ \hrule
\subsection*{(g): }
Using the Rayleigh quotient iteration: 
\begin{align*}
\sigma & = \frac{{\bf x}^T {\bf A}{\bf x}}{{\bf x}^T {\bf x}}\\
\implies \sigma & = \frac{7}{2} = 3.5
\end{align*}\hrule
\subsection*{(h): }
The inverse iteration would converge to the eigenvector corresponding to the smallest (in absolute value) eigenvalue of $\bf A$ and hence it would converge to the eigenvector corresponding to $\lambda = -1$. \\ \hrule
\subsection*{(i): }
Inverse-iteration with shift $=2$ would return the eigenvalue closest to the shift, i.e. 3. \\ \hrule
\subsection*{(j): }
Since the matrix is general (non-symmetric), using the Q-R iteration would result in A converging to a triangular matrix. 
\end{document}