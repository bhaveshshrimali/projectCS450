\ProvidesPackage{commands}
\documentclass[11pt]{article}
\usepackage{epstopdf}
\usepackage{subfigure,graphicx}
\usepackage{amsmath}
\usepackage{epsf}
\usepackage{amsfonts}
\usepackage{amssymb}
\usepackage{color}
\usepackage{mathtools}
\usepackage{placeins}
\usepackage{booktabs}
\usepackage{enumitem}
\usepackage{caption}
\usepackage[margin=0.8in, paperwidth=8.5in, paperheight=11in]{geometry}
\usepackage{amsfonts}
\usepackage{amsmath}
\usepackage{amsbsy}
\usepackage{authblk}
\usepackage{graphicx}
\usepackage{listings}
\usepackage{array}
\usepackage{titlesec}
\usepackage{amssymb}
\usepackage{bm}
\usepackage{mathtools}
\usepackage{titlesec}

\usepackage[latin1]{inputenc}\newcommand{\bs}[1]{\boldsymbol{#1}}
\newcommand{\del}[2]{\frac{\partial {#1}}{\partial {#2}}}
\newcommand{\D}[2]{\frac{D^{\overline{\alpha}}}{\overline{\alpha !}}{#1}(#2,#2)\ {\bf x}^{\overline{\alpha}}}
\newcommand{\dv}[3]{\frac{{\rm d}^{#1}{#2}}{d{#3}^{#1}}}
\newcommand{\ddel}[5]{\frac{\partial^{ {#1} + {#2}} {#3}}{\partial {#4}^{#1} \partial{#5}^{#2}}}
\newcommand{\dev}{{\rm {\bf dev}}}
\newcommand{\proj}[1]{\frac{1}{R^2}{\bf X}\otimes{\bf X}}
\newcommand{\Ie}[1]{I^{\rm e}_{#1}}
\newcommand{\Ce}[1]{\bf C^{\rm e^{#1}}}
\newcommand{\Fe}[2]{F^{\rm e^{#2}}_{#1}}
\newcommand{\Fv}[2]{F^{\rm v^{#2}}_{#1}}
\newcommand{\f}[2]{f^{\rm {#2}}_{#1}}
\newcommand{\fv}[2]{f^{\rm v^{#2}}_{#1}}
\newcommand{\dfv}[2]{\dot{f}^{\rm v^{#2}}_{#1}}
\newcommand{\tGam}[2]{\tilde{\Gamma}^{\rm v^{#2}}_{#1}}
\newcommand{\Gam}[2]{\Gamma^{\rm v^{#2}}_{#1}}
\newcommand{\A}[1]{\mathcal{A}_{#1}}
\newcommand{\F}[2]{F^{\rm #2}_{#1}}
\newcommand{\hpeq}{\hat{\psi}^{\rm Eq}}
\newcommand{\hpneq}{\hat{\psi}^{\rm NEq}}
\newcommand{\etak}{\eta_K({I_1,I_2,J},{\bf C^{\rm e}, B^{\rm v}})}
\newcommand{\nuk}{\nu_K({I_1,I_2,J},{\bf C^{\rm e}, B^{\rm v}})}
\newcommand{\thetak}{\theta_K({I_1,I_2,J},{\bf C^{\rm e}, B^{\rm v}})}
\newcommand{\etaj}{\eta_J({I_1,I_2,J},{\bf C^{\rm e}, B^{\rm v}})}
\newcommand{\dFv}[2]{\dot{F}^{\rm v^{#2}}_{#1}}
\newcommand{\hatpsi}{\widehat{\psi}(I_1, I_2,I^{\rm e}_1,I^{\rm e}_2,J)}
\newcommand{\hpsi}{\widehat{\psi}(I_1,I^{\rm e}_1,J)}
\newcommand{\Fh}[1]{\widehat{\mathcal{F}}\left({\bf F, \Fv{}{}}, {#1}\right)}
\newcommand{\Fhstar}[1]{\widehat{\mathcal{F}}^*\left({\bf F, \Fv{}{}}, {#1}\right)}
\newcommand{\sbar}{\overline{\bm{\sigma}}}
\newcommand{\hpsicomp}[1]{\sum_{r=1}^{2}\left\{\frac{3^{1-\alpha_r}}{2\alpha_r}\mu_r(I^{\alpha_r}_1-3^{\alpha_r})
+\frac{3^{1-a_r}}{2a_r}m_r({\Ie{1}}^{^{a_r}}-3^{a_r})\right\}
+\mu{#1}+\kappa{#1}^2}
\newcommand{\Ni}[1]{N^{(e)}_i(#1)}
\newcommand{\hNi}[1]{\hat{{N}}^{(e)}_i(#1)}
\newcommand{\Ld}{L^{\dagger}}
\newcommand{\intinfinf}{\int_{-\infty}^{\infty} \int_{-\infty}^{\infty}}
\newcommand{\LLnorm}[1]{\left\lVert{#1}\right\rVert_2}
\newcommand{\Linorm}[1]{{\left\lVert{#1}\right\rVert_\infty}}
\newcommand{\tr}{\rm tr}
\newcommand{\deldel}[2]{\frac{\partial^2 {#1}}{\partial {#2}^2}}
\newcommand{\kd}[1]{\delta_{#1}}



\titlespacing\section{10pt}{10pt plus 4pt minus 2pt}{10pt plus 2pt minus 2pt}
\titlespacing\subsection{0pt}{8pt plus 4pt minus 2pt}{8pt plus 2pt minus 2pt}
\titlespacing\subsubsection{0pt}{12pt plus 4pt minus 2pt}{6pt plus 2pt minus 2pt}
\titlespacing*{\title}{-2ex}{*-2ex}{-2ex}
\usepackage{color} %red, green, blue, yellow, cyan, magenta, black, white
\definecolor{mygreen}{RGB}{28,172,0} % color values Red, Green, Blue
\definecolor{mylilas}{RGB}{170,55,241}
\setlength\parindent{0pt}
\graphicspath{{/}}

\title{\bf CSE 401: Numerical Analysis \\ HW 3}
\author{Bhavesh Shrimali \\ NetID: bshrima2}
\date{\today}
\titlespacing*{\title}{-2ex}{*-2ex}{-2ex}
\begin{document}
\maketitle \hrule \hrule \hrule
\section*{Solution (4): }
Given Equation
\[
x^2-3x+2
\]
Corresponding choices of functions for fixed point iterations: 
\begin{align*}
g_1(x) & = (x^2+2)/3 \\
g_2(x) & = \sqrt{3x-2}\\
g_3(x) & = 3-2/x\\
g_4(x) & = (x^2-2)/(2x-3)
\end{align*}
\subsection*{(a):}
\begin{align*}
g_1(x) 
=
(x^2+2)/3\ ; \ \ \ g'_1(x) = \frac{2x}{3}\ ; \ \ \ |g'_1(2)| = \frac{4}{3} \nless 1
\end{align*}
Thus the fixed point iteration corresponding to this function, diverges. \\ \hrule
\subsection*{(b):}
\begin{align*}
g_2(x) 
=
\sqrt{3x-2}\ ; \ \ \ g'_2(x) = \frac{3}{2\sqrt{3x-2}}\ ; \ \ \ |g'_2(2)| = \frac{3}{4} < 1
\end{align*}
Since the derivative of the function, $g'_2(x)$, does not vanish at $x=2$, it has a linear convergence rate with the constant $C=3/4$. \\ \hrule
\subsection*{(c): }
\begin{align*}
g_3(x)
=
3-2/x\ ; \ \ \ g'_3(x) = \frac{2}{x^2}\ ; \ \ \ |g'_3(2)| = \frac{1}{2} < 1
\end{align*}
Again, in this case the derivative of the function $g_3(x)$, evaluated at 2, does not vanish and hence the convergence rate is linear with $C=\frac{1}{2}$. \\ \hrule
\subsection*{(d): }
\begin{align*}
g_4(x) & = (x^2-2)/(2x-3)\ ; \ \ \ g'_4(x) = \frac{2x^2-6x+4}{{2x-3}^2} = \frac{2 (x-2)(x-1)}{{2x-3}^2}\ ; \ \ \ |g'_4(2)| = 0 
\end{align*}
Here the derivative of the function $g_4(x)$ vanishes at $x=2$, and hence  it ensures that there is at-least quadratic convergence. 
\subsubsection*{Note: }
It is assumed that the starting guess, for Newton's Method, is close enough for the above asymptotic convergence rates to hold valid. \\
\hrule \hrule \hrule
\end{document}