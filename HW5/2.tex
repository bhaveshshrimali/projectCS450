\input{commands}
\title{\bf CSE 401: Numerical Analysis \\ HW 5}
\author{Bhavesh Shrimali \\ NetID: bshrima2}
\date{\today}
\titlespacing*{\title}{-2ex}{*-2ex}{-2ex}
\begin{document}
\maketitle \hrule \hrule \hrule
\section*{Solution 2:}
Given data points 
\begin{align*}
(-1,1),(0,0),(1,1)
\end{align*}
\subsection*{Using Monomial Basis function: }
The monomial basis functions are given by the following formula
\begin{align*}
\phi_k(x) = x^{k-1} \implies \phi_1(x)  = 1\ ;\ \ \phi_2(x)  = x\ ;\ \ \ \phi_3(x)  = x^2\ \ \ \  
\end{align*}
And the corresponding Vandermonde matrix is given by
\begin{align*}
{\bf V} = 
\begin{bmatrix}
1 & -1 & 1 \\
1 & 0 & 0 \\
1 & 1 & 1
\end{bmatrix}
\end{align*}
Let the corresponding weights be denoted by $w_1$, $w_2$, $w_3$ and therefore we get the corresponding linear system 
\begin{align*}
\begin{bmatrix}
1 & -1 & 1 \\
1 & 0 & 0 \\
1 & 1 & 1
\end{bmatrix}
\begin{Bmatrix}
w_1 \\ w_2 \\ w_3
\end{Bmatrix} = 
\begin{bmatrix}
1 \\ 0 \\ 1
\end{bmatrix}
\implies 
\begin{Bmatrix}
w_1 \\ w_2 \\ w_3
\end{Bmatrix} =
\begin{bmatrix}
0\\0\\1
\end{bmatrix}
\end{align*}
Hence the interpolating polynomial 
\begin{align*}
\boxed{m(x) = x^2}
\end{align*}\hrule


\subsection*{Using Lagrange Basis functions: }
The lagrange basis functions are given by the following formula
\begin{align*}
\phi_k(x) = \prod_{j=1,j \neq k}^{j=n} \frac{x-x_j}{x_k-x_j} \implies \phi_1(x)  = \frac{(x)(x-1)}{2}\ ;\ \ \phi_2(x)  = (x+1)(x-1)\ ;\ \ \ \phi_3(x)  = \frac{(x+1)(x)}{2}\ \ \ \  
\end{align*}
And the corresponding Coefficient matrix is given by
\begin{align*}
{\bf V} = 
\begin{bmatrix}
1 & 0 & 0 \\
0 & 1 & 0 \\
0 & 0 & 1
\end{bmatrix}
\end{align*}
Let the corresponding weights be denoted by $w_1$, $w_2$, $w_3$ and therefore we get the corresponding linear system 
\begin{align*}
\begin{bmatrix}
1 & 0 & 0 \\
0 & 1 & 0 \\
0 & 0 & 1
\end{bmatrix}
\begin{Bmatrix}
w_1 \\ w_2 \\ w_3
\end{Bmatrix} = 
\begin{bmatrix}
1 \\ 0 \\ 1
\end{bmatrix}
\implies 
\begin{Bmatrix}
w_1 \\ w_2 \\ w_3
\end{Bmatrix} =
\begin{bmatrix}
0\\0\\1
\end{bmatrix}
\end{align*}
Hence the interpolating polynomial 
\begin{align*}
\boxed{m(x) = \frac{(x-1)(x)}{2} + \frac{(x+1)(x)}{2} = x^2}
\end{align*}\hrule




\subsection*{Using Newton Basis functions: }
The newton basis functions are given by the following formula
\begin{align*}
\phi_k(x) = \prod_{j=1}^{k-1} (x-x_k) \implies \phi_1(x)  = 1\ ;\ \ \phi_2(x)  = x+1\ ;\ \ \ \phi_3(x)  = (x)(x+1)\ \ \ \  
\end{align*}
And the corresponding Coefficient matrix is given by
\begin{align*}
{\bf V} = 
\begin{bmatrix}
1 & 0 & 0 \\
1 & 1 & 0 \\
1 & 2 & 2
\end{bmatrix}
\end{align*}
Let the corresponding weights be denoted by $w_1$, $w_2$, $w_3$ and therefore we get the corresponding linear system 
\begin{align*}
\begin{bmatrix}
1 & 0 & 0 \\
1 & 1 & 0 \\
1 & 2 & 2
\end{bmatrix}
\begin{Bmatrix}
w_1 \\ w_2 \\ w_3
\end{Bmatrix} = 
\begin{bmatrix}
1 \\ 0 \\ 1
\end{bmatrix}
\implies 
\begin{Bmatrix}
w_1 \\ w_2 \\ w_3
\end{Bmatrix} =
\begin{bmatrix}
1\\-1\\1
\end{bmatrix}
\end{align*}
Hence the interpolating polynomial 
\begin{align*}
m(x) & = 1\cdot1 -1\cdot(x+1)+1\cdot(x^2+x) = 1-x-1+x^2+x 
\\ & \boxed{ = x^2}
\end{align*}\hrule
\subsection*{Remarks: }
It can be seen that all of the above basis functions give a unique polynomial. This follows from the fact that the interpolant of degree $n-1$ to $n$ data points is unique and hence all the three methods should give a unique polynomial. Thus for '3' points a polynomial of degree '2' should be unique, and this is verified above. 
\end{document}