\ProvidesPackage{commands}
\documentclass[11pt]{article}
\usepackage{epstopdf}
\usepackage{subfigure,graphicx}
\usepackage{amsmath}
\usepackage{epsf}
\usepackage{amsfonts}
\usepackage{amssymb}
\usepackage{color}
\usepackage{mathtools}
\usepackage{placeins}
\usepackage{booktabs}
\usepackage{enumitem}
\usepackage{caption}
\usepackage[margin=0.8in, paperwidth=8.5in, paperheight=11in]{geometry}
\usepackage{amsfonts}
\usepackage{amsmath}
\usepackage{amsbsy}
\usepackage{authblk}
\usepackage{graphicx}
\usepackage{listings}
\usepackage{array}
\usepackage{titlesec}
\usepackage{amssymb}
\usepackage{bm}
\usepackage{mathtools}
\usepackage{titlesec}

\usepackage[latin1]{inputenc}\newcommand{\bs}[1]{\boldsymbol{#1}}
\newcommand{\del}[2]{\frac{\partial {#1}}{\partial {#2}}}
\newcommand{\D}[2]{\frac{D^{\overline{\alpha}}}{\overline{\alpha !}}{#1}(#2,#2)\ {\bf x}^{\overline{\alpha}}}
\newcommand{\dv}[3]{\frac{{\rm d}^{#1}{#2}}{d{#3}^{#1}}}
\newcommand{\ddel}[5]{\frac{\partial^{ {#1} + {#2}} {#3}}{\partial {#4}^{#1} \partial{#5}^{#2}}}
\newcommand{\dev}{{\rm {\bf dev}}}
\newcommand{\proj}[1]{\frac{1}{R^2}{\bf X}\otimes{\bf X}}
\newcommand{\Ie}[1]{I^{\rm e}_{#1}}
\newcommand{\Ce}[1]{\bf C^{\rm e^{#1}}}
\newcommand{\Fe}[2]{F^{\rm e^{#2}}_{#1}}
\newcommand{\Fv}[2]{F^{\rm v^{#2}}_{#1}}
\newcommand{\f}[2]{f^{\rm {#2}}_{#1}}
\newcommand{\fv}[2]{f^{\rm v^{#2}}_{#1}}
\newcommand{\dfv}[2]{\dot{f}^{\rm v^{#2}}_{#1}}
\newcommand{\tGam}[2]{\tilde{\Gamma}^{\rm v^{#2}}_{#1}}
\newcommand{\Gam}[2]{\Gamma^{\rm v^{#2}}_{#1}}
\newcommand{\A}[1]{\mathcal{A}_{#1}}
\newcommand{\F}[2]{F^{\rm #2}_{#1}}
\newcommand{\hpeq}{\hat{\psi}^{\rm Eq}}
\newcommand{\hpneq}{\hat{\psi}^{\rm NEq}}
\newcommand{\etak}{\eta_K({I_1,I_2,J},{\bf C^{\rm e}, B^{\rm v}})}
\newcommand{\nuk}{\nu_K({I_1,I_2,J},{\bf C^{\rm e}, B^{\rm v}})}
\newcommand{\thetak}{\theta_K({I_1,I_2,J},{\bf C^{\rm e}, B^{\rm v}})}
\newcommand{\etaj}{\eta_J({I_1,I_2,J},{\bf C^{\rm e}, B^{\rm v}})}
\newcommand{\dFv}[2]{\dot{F}^{\rm v^{#2}}_{#1}}
\newcommand{\hatpsi}{\widehat{\psi}(I_1, I_2,I^{\rm e}_1,I^{\rm e}_2,J)}
\newcommand{\hpsi}{\widehat{\psi}(I_1,I^{\rm e}_1,J)}
\newcommand{\Fh}[1]{\widehat{\mathcal{F}}\left({\bf F, \Fv{}{}}, {#1}\right)}
\newcommand{\Fhstar}[1]{\widehat{\mathcal{F}}^*\left({\bf F, \Fv{}{}}, {#1}\right)}
\newcommand{\sbar}{\overline{\bm{\sigma}}}
\newcommand{\hpsicomp}[1]{\sum_{r=1}^{2}\left\{\frac{3^{1-\alpha_r}}{2\alpha_r}\mu_r(I^{\alpha_r}_1-3^{\alpha_r})
+\frac{3^{1-a_r}}{2a_r}m_r({\Ie{1}}^{^{a_r}}-3^{a_r})\right\}
+\mu{#1}+\kappa{#1}^2}
\newcommand{\Ni}[1]{N^{(e)}_i(#1)}
\newcommand{\hNi}[1]{\hat{{N}}^{(e)}_i(#1)}
\newcommand{\Ld}{L^{\dagger}}
\newcommand{\intinfinf}{\int_{-\infty}^{\infty} \int_{-\infty}^{\infty}}
\newcommand{\LLnorm}[1]{\left\lVert{#1}\right\rVert_2}
\newcommand{\Linorm}[1]{{\left\lVert{#1}\right\rVert_\infty}}
\newcommand{\tr}{\rm tr}
\newcommand{\deldel}[2]{\frac{\partial^2 {#1}}{\partial {#2}^2}}
\newcommand{\kd}[1]{\delta_{#1}}



\titlespacing\section{10pt}{10pt plus 4pt minus 2pt}{10pt plus 2pt minus 2pt}
\titlespacing\subsection{0pt}{8pt plus 4pt minus 2pt}{8pt plus 2pt minus 2pt}
\titlespacing\subsubsection{0pt}{12pt plus 4pt minus 2pt}{6pt plus 2pt minus 2pt}
\titlespacing*{\title}{-2ex}{*-2ex}{-2ex}
\usepackage{color} %red, green, blue, yellow, cyan, magenta, black, white
\definecolor{mygreen}{RGB}{28,172,0} % color values Red, Green, Blue
\definecolor{mylilas}{RGB}{170,55,241}
\setlength\parindent{0pt}
\graphicspath{{/}}

\title{\bf CSE 401: Numerical Analysis \\ HW 5}
\author{Bhavesh Shrimali \\ NetID: bshrima2}
\date{\today}
\titlespacing*{\title}{-2ex}{*-2ex}{-2ex}
\begin{document}
\maketitle \hrule \hrule \hrule
\section*{Solution 5: Chebyshev Quadrature}
Give that all the weights are equal. Let the weights be denoted by $w$ and the sampled points be denoted by $x_1$, $x_2$, $x_3$. Therefore we have a total of 4 parameters to be determined. Now for three point Chebyshev quadrature, the resulting rule should integrate polynomials upto degree 3 exactly.   
\subsection*{Constant Polynomial: }
We have
\begin{align}
\int_{-1}^1 1\ dx = 2 = w \left( 1 + 1 + 1 \right) \implies w = \frac{2}{3}
\end{align}
\subsection*{Linear Polynomial: }
We have
\begin{align}
\int_{-1}^1 x\ dx = 0 = w \left( x_1 + x_2 + x_3 \right) \implies  x_1 + x_2 + x_3 = 0
\end{align}
\subsection*{Quadratic Polynomial: }
We have
\begin{align}
\int_{-1}^1 x^2\ dx = \frac{2}{3} = w \left( x^2_1 + x^2_2 + x^2_3 \right) \implies  x^2_1 + x^2_2 + x^2_3 = 1
\end{align}
\subsection*{Cubic Polynomial: }
We have
\begin{align}
\int_{-1}^1 x^3\ dx = 0 = w \left( x^3_1 + x^3_2 + x^3_3 \right) \implies  x^3_1 + x^3_2 + x^3_3 = 0
\end{align}
In order to analytically solve the above four equations we proceed by making use of the algebraic identity:  
\begin{align}
\nonumber x^3_1 + x^3_2 + x^3_3 - 3x_1x_2x_3  & = (x_1+x_2+x_3)(x^2_1+x^2_2+x^2_3 - x_1x_2 - x_2x_3 - x_1x_3)\\
& \implies  x^3_1 + x^3_2 + x^3_3 = 3x_1x_2x_3\ ;\ \ \forall (x_1+x_2+x_3) = 0
\end{align}
Using equation (5) together equation (4) gives
\begin{align}
x_1x_2x_3 = 0\ ;
\end{align}
\newpage This implies the following conditions:
\begin{itemize}
\item $x_1 = 0\ ;x_2 \neq 0\ ; x_3 \neq 0 $
\item $x_2 = 0\ ;x_1 \neq 0\ ; x_2 \neq 0 $
\item $x_3 = 0\ ;x_1 \neq 0\ ; x_2 \neq 0 $
\end{itemize}
or 
\begin{itemize}
\item $x_1\ , \ x_2 = 0\ ; \ \ x_3 \neq 0 $
\item $x_2\ , \ x_3 = 0\ ; \ \ x_1 \neq 0 $
\item $x_1\ , \ x_3 = 0\ ; \ \ x_2 \neq 0 $
\end{itemize} 
This leads to the following conclusions: 
\begin{itemize}
\item Now if any of the above three equations are satisfied then by equation (2) the remaining third variable would also be equal to zero and thus equation (3) would not hold true. Thus the only possibility is that only one of the sample points is zero. 
\item Note that due to the symmetry of the problem in $x_1$, $x_2$ and $x_3$ we can see that the solution corresponding to the first case, i.e.  $x_1 = 0\ ;x_2 \neq 0\ ; x_3 \neq 0 $ would be equivalent to the other two cases. Therefore we present the solution corresponding to the first case. 
\end{itemize}
\begin{align*}
x_1 = 0\ ; \ \ \ x_2 \neq 0\ ; \ \ \ x_3 \neq 0  
\end{align*}
Then by (2) and (3)
\begin{align*}
x_2 = -x_3\ ; \ \ \ \ 2x^2_3 = 1 \implies x_3 = \pm \frac{1}{\sqrt{2}}\ ; \ \ \ x_2 = \mp \frac{1}{\sqrt{2}}
\end{align*}
Again due to the symmetry of the problem in $x_2$ and $x_3$ it is sufficient to consider only one of the above possibilities as the other one is equivalent
\begin{align*}
x_2 = \frac{1}{\sqrt{2}}\ ; \ \ \ \ x_3 = -\frac{1}{\sqrt{2}}
\end{align*}
Thus the solution of the system of nonlinear equations is given by 
\begin{align}
\boxed{w = \frac{2}{3}\ ; \ \ \ x_1 = 0\ ; \ \ \ x_2 = \frac{1}{\sqrt{2}}\ ; \ \ \ x_3 = -\frac{1}{\sqrt{2}}}
\end{align}\hrule
\subsection*{Remarks: }
Since the above rule is for three points, it can integrate any polynomial up to and including degree 3 exactly. The degree of a quadrature rule, by definition, is the largest integer, {\bf n}, such that 
\begin{align*} 
\text{the rule exactly integrates}\ ,\ x^k\ ,\forall k \in [0,n] 
\end{align*}
Thus the degree of the resulting rule is three. ({\bf n = 3}). This can be verified by integrating a higher order polynomial, say degree 4 and verify if the quadrature rule works. 
\begin{align}
\int_{-1}^1 x^4\ dx = \frac{2}{5}\ ; \ \ \ \text{but}\ \ \ \frac{2}{3}\left( \frac{1}{4} + \frac{1}{4}\right) = \frac{1}{3} 
\end{align} 
Hence the quadrature rule doesn't work. Thus the degree is 3. \\ \hrule\hrule\hrule
\end{document}