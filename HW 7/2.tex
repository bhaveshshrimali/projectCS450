\input{commands}
\title{\bf CSE 401: Numerical Analysis \\ HW 7}
\author{Bhavesh Shrimali \\ NetID: bshrima2}
\date{\today}
\titlespacing*{\title}{-2ex}{*-2ex}{-2ex}
\begin{document}
\maketitle \hrule\hrule\hrule
\section*{Solution 2: }
Given ODE
\begin{align*}
u'' 
=
u^3 +t\ \ ; \ \ \ \ \ \ a<t<b
\end{align*}
subject to boundary conditions
\begin{align*}
u(a) = \alpha\ \ ; \ \ \ \ \ \ u(b) = \beta
\end{align*}\hrule
\subsection*{(a):}
The Euler's method for an ODE 
\begin{align*}
{\bf y}' = {\bf f}(t,{\bf y})
\end{align*}
is illustrated via the iterative scheme given below:
\begin{align}
{\bf y}_{i+1} 
=
{\bf y}_i
+ h\cdot{\bf f}(t_i,{\bf y}_i)
\label{Eq1}
\end{align}
Let $u'(a) = s$, hence for a single step 
\begin{align*}
& h=b-a\ \ ; \ \ \ \ \ \ \ t_i = a\ \ ; \ \ \ \ \ \  {\bf y}_0 = \begin{Bmatrix}\alpha\\s \end{Bmatrix}; \ \ \ \ \ {\bf y}_{1} = \begin{Bmatrix}\beta\\u'_{1} \end{Bmatrix}
\end{align*}
Rewriting the system of equations in the iterative form given by 
(\ref{Eq1})
\begin{align*}
\begin{bmatrix}
\beta\\u'_{1}
\end{bmatrix}
=
\begin{Bmatrix}
\alpha\\s
\end{Bmatrix}
+ (b-a)\cdot
\begin{Bmatrix}
s\\a+\alpha^3
\end{Bmatrix}
\end{align*}
Thus the algebraic relation for the initial slope, to be determined, is given by:  
\begin{align*}
\boxed{
 \bf {\bm\beta} = {\bm\alpha} + (b-a) s
}
\end{align*}
\subsection*{(b):}
The initial value of the slope obtained from a single step of Euler's method is the slope of the {\bf straight-line} joining the end points (the independent variable ($t$) and the value of the solution at those points $u(t)$ of the domain. 
\begin{align*}
\boxed{
{\bf s}
=
\frac{\bm \beta - \bm \alpha}{\bf b -a}
}
\end{align*}
\end{document}